



\Large
\textbf{\color{verydarkred}MAS2903: Introduction to  Bayesian Methods}


\large
\textbf{Dr. Matthew Fisher}


\normalsize

Office: \textsf{Room 4.14a, Herschel Building}



Email: \texttt{matthew.fisher@newcastle.ac.uk} 






\normalsize












\noindent \textbf{\color{verydarkred}Assessment}

MAS2903 is a 10 credit module assessed via a formal written examination in the summer (likely a traditional 2 hour written paper, taken in-person under exam conditions) and some in-course assessment.  The make-up is 80\% summer exam, 20\% in-course assessment.  Further details:
\begin{itemize}
\item Summer exam, 100 marks in total
\begin{itemize}
\item Section A (40 marks): Consisting mainly of short, straightforward, data-response style questions; perhaps with some multiple choice elements
\item Section B (60 marks): Longer, harder questions, requiring some contextual discussion and interpretation.  Questions more mathematically challenging
\item Answer \textit{all} questions in \textit{both} parts
\end{itemize}

\item In-course assessment, 80 marks in total
\begin{itemize}
\item Problem Solving Exercises 1 (20 marks)
\item Numbas Assessment 1 (20 marks)
\item Problem Solving Exercises 2 (20 marks)
\item Numbas Assessment 2 (20 marks)
\end{itemize}
\end{itemize}
 Some of the questions in the in-course assessment will require you to produce plots in \texttt{R}, so it might be a good idea to type up your solutions and upload a single PDF file (which includes any necessary plots/computing work) to Canvas.

\clearpage
\noindent \textbf{\color{verydarkred}Books and other resources}

At various points in the module you will be directed to additional reading, should you want extra examples or just another source of information relating to the material presented in the notes.  This will primarily be text books (many of which are available to view online), but perhaps other sources (e.g. journal publications, magazine articles etc.).  

\noindent Some useful books include:
\begin{itemize}
\item ``\textit{Bayes Rule: A Tutorial Introduction to Bayesian Analysis}'' - James Stone
\item ``\textit{Doing Bayesian Data Analysis: A Tutorial with R, JAGS, and Stan}'' - John Krushke
\item ``\textit{Bayesian Statistics: An Introduction}'' - Peter Lee
\end{itemize}
``Bayes rule'' is a good introduction to the main concepts in Bayesian statistics but doesn't cover everything in this course.
The other books are broader references which go well beyond the contents of this course.

\noindent \textbf{\color{verydarkred}Acknowledgements}

These notes are a small modification on notes by Dr Lee Fawcett.


\chapter*{Preface: What is Bayesian Statistics?}

There are two main approaches to statistics: \emph{frequentist} (or classical) statistics and \emph{Bayesian} statistics.
All of the statistics teaching you've encountered so far is likely to be about frequentist methods.
Bayesian methods are substantially different and can feel quite strange to start with.
So, before starting the main course material, this \emph{non-examinable} preface explains the concepts behind Bayesian statistics and how they differ from frequentist approaches.

One way of defining statistics is as a way to learn about the world from some data which is subject to random variation.
For example, in \emph{climate science} we want to learn about the climate given some imperfect measurements.
Some statistical questions that we might ask in this field are:
\begin{itemize}
\item What is our best estimate of the world temperature this year?\\
This is referred to as a \emph{point estimation} problem.
\item What is a plausible range of values?\\
This is referred to as an \emph{interval estimation} (or uncertainty quantification) problem.
\item What will the temperature be in 100 years?\\
This is a \emph{prediction} problem.
\item Is the climate warming?\\
This is a \emph{hypothesis testing} problem.
\end{itemize}

Frequentist and Bayesian methods address all these types of problem: point estimation, interval estimation, prediction and hypothesis testing.
But they use different approaches to do so.
Some typical frequentist approaches include:
\begin{itemize}
\item Least squares (point estimation)
\item Maximum likelihood (point estimation)
\item Confidence intervals (interval estimation)
\item Test statistics and p-values (hypothesis testing)
\end{itemize}
Bayesian statistics doesn't use any of these familiar methods!
However, there are some connections between them and their Bayesian alternatives which will be explored later in the course.
For example we'll see that the idea of the \emph{likelihood function} is very important in Bayesian statistics.

\section*{Motivating example}

Air France Flight 447 disappeared over the Atlantic Ocean on June 1st 2009,
during an overnight flight from Rio de Janeiro to Paris.
A search and rescue mission was launched the following morning.
After five days some floating wreckage was found, but there was no sign of the ``black box'' recorders which would reveal what had happened to the flight, and the search stalled.

The black boxes could be anywhere within an area of the South Atlantic the size of Switzerland (6,500 square miles).
The ``mid-Atlantic ridge'' on the ocean floor lies between two tectonic plates and is just as mountainous as Switzerland!
It's also so remote that scientists have not yet charted the sea-bed.
The search method used was sonar-detectors, emitting sound waves which would bounce back once they hit something.
However, after two years of meticulously searching an area north of the plane’s trajectory (after analysing debris drift), nothing had been found.

Metron, Inc., of Retson, Virginia had been performing Bayesian analyses to help the search effort.
Included in their analysis were:
\begin{enumerate}
\item
Data from 9 previous airline crashes involving loss of pilot
control - reduced search area to 1,600 square miles.
\item
Expert opinions on the credibility of the flight data.
\item
Expert opinions about whether or not the black box
`pingers' might have become damaged on impact.
\item
Positions/recovery times of bodies found drifting - expert
opinions assigned to the reliability of this data because of
the turbulent equatorial waters.
\item
Expert information from oceanographers: Sea state,
visibility, underwater geography,\ldots
\end{enumerate}
All were combined through Bayes Theorem to give a probability map of the most likely locations to search.
In April 2011 they decided to try assuming that the `pingers' in item 3 were probably damaged.
One week later the black boxes were found!\footnote{
For more details see \texttt{https://www.technologyreview.com/s/527506/how-statisticians-found-}
\texttt{air-france-flight-447-two-years-after-it-crashed-into-atlantic/}.
}

\section*{The frequentist vs. Bayesian debate}

This example illustrates some key features of the Bayesian approach:
\begin{itemize}
\item It incorporates expert opinion.
\item It outputs a probability distribution over the possible choices.
\item The calculations required are all based on Bayes Theorem.
\end{itemize}
In contrast, in frequentist statistics, it is difficult (but not impossible) to incorporate opinions, and the output is not in the form of a probability distribution.
For example a 95\% frequentist confidence interval is a range which will contain
the true value 95\% of the time if the analysis is repeated
a large number of times for different data obtained under
the same process\footnote{
Here 95\% is the \emph{long-run frequency} that the interval contains the correct value.
This kind of property is where the name ``frequentist'' comes from.
}.
In contrast a Bayesian 95\% interval estimate (see Chapter 4)
is a range with a 95\% probability of containing the true value.
This Bayesian version is arguably much easier to interpret.

There has been a decades-long argument about whether Bayesian or frequentist methods are better.
Both sides have some good points.
For example, there is the argument we've just made about Bayesian interval estimates being easy to interpret.
On the other hand one of the main arguments against the Bayesian approach is that it is \emph{subjective}.
This is because its results are based on expert opinions, and if we got a different opinion from another expert our results would change even though the data are the same!
Bayesians have counter-arguments.



In practice many problems are better suited to be tackled with methods from one approach or the other.
For example Bayesian statistics was well suited to the Air France example as incorporating expert knowledge was particularly important in the solution.
Therefore it's good to be familiar with both Bayesian and frequentist methods.

\section*{The history of Bayesian statistics}

The reason why we use the word ``Bayesian'' is because Bayes Theorem is crucial for statistical analysis if we adopt the Bayesian approach.
Thomas Bayes (1702--1761; see picture below) was a Presbyterian minister in Tunbridge Wells, Kent.
Bayes solution to a problem in probability theory was presented in the \textit{Essay towards Solving a Problem in the Doctrine of Chances}, published posthumously by his friend Richard
Price in the \textit{Philosophical Transactions of the Royal Society of London}.
This paper gave us Bayes Theorem.
Its modern form is due to Pierre-Simon Laplace who generalised Bayes work and produced the first theory of Bayesian inference, which was referred to as ``inverse probability''.  Laplace's theory had weaknesses\footnote{For example it relied mainly on \emph{flat priors}, which will be discussed and criticised in Chapter 3.} and in the early 20th century it was replaced by the more rigorous frequentist approach.  R.~A.~Fisher was one of the leading creators of this approach.
He was also the first to use the term ``Bayesian statistics''.
\begin{figure}

\includegraphics{images/bayes1.eps} \hspace{0.25cm} \includegraphics{images/bayes_essay.eps}

\caption{Portrait believed to be of Thomas Bayes, and letter from Richard Price to the Philosophical Transactions of the Royal Society of London}
\end{figure}


\begin{figure}[ht]

\includegraphics{images/Laplace.eps} \includegraphics{images/figures/Fisher.eps}

\caption{Pierre-Simon Laplace (left) and R.~A.~Fisher (right)}
\end{figure}

Nonetheless there was still some interest in Bayesian methods.
For example, Alan Turing's group invented some new Bayesian methods to help break the Enigma code in World War II.
Before and, increasingly, after the war a few researchers worked on developing a rigorous theory of Bayesian statistics.

Bruno de Finetti (1906--1985) was an Italian probabilist who developed his ideas on \emph{subjective probability}\footnote{Covered in Chapter 1.} from the 1920s onwards, drawing upon ideas from H. Jeffreys, I.J. Good (one of the Enigma codebreakers) and B.O. Koopman.
His classic book on the topic is \textit{The Theory of Probability} (1974).


Dennis Lindley (1923--2013) was a leading British Bayesian.
In his early career, he worked to find a mathematical basis for the subject of statistics.
In 1954, Lindley met Leonard Savage and both found a deeper justification for statistics in Bayesian theory, turning into critics of the classical statistical inference they had hoped to justify.


Many of these theoretical advances are covered in this course.
However, Bayesian statistics was still little used in practice.
The difficulty was that applying Bayes theorem often became very mathematically challenging,
resulting in problems that were too difficult or impossible to solve by hand.
\subsection*{Recent developments}
In the 1990s a solution to these mathematical problems was developed, making use of the emergence of more powerful computers and ``Markov chain Monte Carlo'' (MCMC) numerical algorithms.
If you take 3rd and 4th year Bayesian courses you will be introduced to these techniques.
MCMC, and subsequent developments, have revolutionised the use of Bayesian Statistics, to
the extent that today Bayesian data analyses are as popular as frequentist approaches and are routinely used in fields as diverse as artificial intelligence, biology, astrophysics and sociology.

\subsection*{Newcastle's contribution}
Since 1980, the number of academic staff in Mathematics & Statistics at Newcastle publishing advanced research using Bayesian methods has increased dramatically. In the 1980s, there was only one Bayesian at Newcastle. Now there are at least 12.
Our Bayesian research topics include environmental extremes, genetics, and modelling biological systems at a molecular level.
